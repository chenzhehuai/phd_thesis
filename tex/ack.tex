%# -*- coding: utf-8-unix -*-
\begin{thanks}

衷心感谢我的导师俞凯教授,正是您对我在学术上的精心指导,项目上的大量锻炼,
生活上的亲切关怀,使我在博士的五年能够不断进步。博士期间是一个人形成自己的
处世哲学和思维方式的重要时光,感谢您对我长久的潜移默化。更要感谢您告诉我, 
“Your destiny is to make life beautiful (你们的使命是让世间更美好)”, 
这成为了我长久的信条,将伴我在未来科学的道路上继续前行。

同时感谢我的导师钱彦旻教授对我的悉心指导。从学术研究到日常生活,钱老师都给了我很多宝
贵的指导和建议。在我的研究遇到困难时,和您的讨论也往往会带来许多启发和
灵感。感谢茅懿琼老师不厌其烦地为我操持着学校的一切,不辞辛劳,不求回报,
为我免去后顾之忧安心科研。感谢吴科老师和吴梦玥老师对我在科研上的点滴帮助。

感谢思必驰信息科技有限公司的语音团队对我长久以来的帮助,谢谢你们为我提供了
前沿的研究课题,广阔的应用场景,和充足的研究资源。科学创新往往与实践相伴,
感激博士期间能够与你们共同进步,也祝贺你们一步步成长为国内顶尖的语音团队。

Sincere gratitude to my mentors during internship: Jasha Droppo, Jinyu Li, Wayne Xiong, Daniel
Povey, Sanjeev Khudanpur, Justin Luitjens, Mahaveer Jain, Yongqiang Wang, Christian Fuegen, Mike Seltzer and Geoffrey Zweig. Special thanks to Daniel Povey: Your loves to speech community always touch me. Your long-term contribution to the open-sourced community will always affect me. 感谢这期间所遇到的合作者和有趣的玩伴们:张晓辉,许海南,王一鸣,吕航,李可,顾骏程,邱航,苗宏宇,陈剑航,你们伴我度过了国外孤独的时光。

感谢一起在上海交大语音实验室学习和工作的师兄师姐及学弟学妹们,特别是在项目和
研究中共同奋斗过的:邓威、游永彬、谭天、陈露、朱苏、童思博、陈博、项煦、万一、郑达、庄毅萌、
尹茂帆、吴越、刘奇、王帅、丁翰林、常烜恺、李豪、陈宽、黄明坤、卢怡宙、李晨达,
感激与你们合作的时光,谢谢你们伴我成长。


感谢本科母校的刘玉老师,您的信任和关怀让我心中每时每刻都感到温暖,您对创业和创新的激情时刻鼓舞着我。大学时代是一个人形成人生观的最主要时期,而Dian团队就是我全部的大学时光。您梦想中的点石大楼已经在我们每一个学子心中落成,创新创业的思想将永远驰骋。

最后,感谢我的父母、爷爷奶奶、外公外婆,我的岳父岳母,谢谢你们对我的爱和支持。将最大的感谢送与我的母亲李子蓉,和太太刘昱,谢谢你们给予我对这个世界的无尽勇气与动力!

\end{thanks}
