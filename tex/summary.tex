%# -*- coding: utf-8-unix -*-
%%==================================================
%% conclusion.tex for SJTUThesis
%% Encoding: UTF-8
%%==================================================

\chapter{全文总结}
\label{chap:sum}

本论文围绕基于深度序列模型的解码搜索技术展开了一系列探索和研究。
本论文的主要贡献在第\ref{chap:gpu}章、第\ref{chap:lsd}章、第\ref{chap:unify}章、第\ref{chap:kws}章中介绍。第一个贡献是提出并行的解码搜索算法并在GPU上实现开源该套算法,具体将在第\ref{chap:sum-gpu}章总结;第二个贡献是基于端到端建模,系统地提出了标签同步算法,具体将在第\ref{chap:sum-lsd}章总结。上述两项贡献从不同层面使得解码搜索技术得到了极大幅度加速。第三个贡献是提出将不同应用中的语音识别推理过程和置信度统一到同一框架中,具体将在第\ref{chap:sum-unify}章总结;第四个贡献是系统地提出针对深度序列模型的序列鉴别性训练框架,具体将在第\ref{chap:sum-kws}章总结。
除上述贡献外,作者在端到端建模和鲁棒语音识别建模中均有贡献,概述于第\ref{chap:intro2-e2e}章和第\ref{chap:intro2-pit}章。可能的后续工作展望在第\ref{chap:sum-future}章中总结。

\section{基于GPU并行计算的搜索速度优化}
\label{chap:sum-gpu}

在本论文中,我们针对Kaldi开源工具包的推理搜索部分进行了一项重要扩展,以便使它能够支持图形处理芯片(GPU)上的WFST解码推理。该框架可以显著加速现有推理算法,特别是在基于深度序列学习的一系列模型上进行了验证。

这是一个通用的离线解码器,该解码器对语言模型和声学模型没有特别的限制,并且可以工作在各种架构的GPU上。
为了支持第二遍重打分和更丰富的后处理,我们的设计基于WFST解码和词图生成的架构~\cite{povey2012generating}。
针对设计中的几大难点,我们提出了如下解决方案:
我们将维特比算法中的令牌合并操作实现为一个GPU并行计算中的原子操作,以便减少维特比束剪枝算法中同步消耗;我们提出了动态负载均衡的方式以更高效地进行并行计算,提高其多线程之间的利用率;我们重新设计了基于GPU并行计算的精确的词图生成和剪枝算法,以便充分利用GPU的性能特点。


在Switchboard 上实验表明,我们所提出的方法在取得完全一致的1-best和词图质量情况下,可以得到3-15倍的加速,并在绝大部分GPU架构上进行了验证。除此之外,如果再进行多句子的并行处理,最终的加速比将达到46倍。
同时我们对这项工作进行了开源~\footnote{\url{https://github.com/chenzhehuai/kaldi/tree/gpu-decoder}},
它将与大多数Kaldi脚本相兼容。这项工作作为Kaldi工具包的一个扩展~\cite{povey2011kaldi},完整实现了基于GPU并行计算的WFST解码。



\section{基于标签同步解码的搜索空间优化}
\label{chap:sum-lsd}

基于端到端建模,
我们系统地提出了标签同步算法,其通过一系列方法使得搜索解码过程从逐帧同步变为标签同步,这包括使用高效的blank结构和后处理方法。该文提出的一系列通用方法在隐马尔科夫模型和连接时序模型上得到了验证。同时我们还介绍了将标签同步算法应用于序列到序列的端到端模型的方案,使之取得了更快和更好的模型收敛和模型准确度。


自动语音识别等序列标注任务的一个独特点是其对相邻帧的时序序列关联性建模。用于对相邻帧进行时序建模的主流序列模型包括隐马尔科夫模型(Hidden Markov Model, HMM)和连接时序模型(Connectionist Temporal Classification, CTC)。针对这些模型,当前主流的推理方法是帧层面的维特比束搜索算法,该算法复杂度很高,限制了语音识别的广泛应用。深度学习的发展使得更强的上下文和历史建模成为可能。通过引入blanks(“空”)单元,端到端建模系统能直接预测标签在给定特征下的后验概率。

我们提出将特征层面的搜索过程改变为标签层面,即搜索空间是由不同历史的标签组成的,使得解码速率等于标注速率,从而小于特征速率。具体来说,在标签推理阶段,对帧层面声学模型的输出增加一步后处理过程:i)判断当前帧是否存在标签输出;ii)若有,执行搜索过程;若无,则丢弃标签输出。因此该后处理过程可被看作是每个输出标签概率计算的近似。与传统方法相比,该方法的优势是搜索空间更小,且搜索过程被大大加速。
我们提出的一系列通用方法在隐马尔科夫模型和连接时序模型上得到了验证。
%
同时本文还研究了将标签同步算法应用于序列到序列的端到端模型的方案。我们使用模块化训练的思想来改善端到端模型建模,使其更易于使用外在知识源来训练每一个端到端模型的子模块。值得注意的是,模型最后需要进行联合优化,因此最终在推理阶段,模型仍然工作在端到端模式下。

在实验部分,我们提出的系统一方面取得大幅度语音识别解码速度改善,另一方面在端到端建模上取得了更快和更好的模型收敛和模型准确度。


\section{基于标签同步解码的统一解码框架}
\label{chap:sum-unify}
在本论文中,我们提出一系列针对不同应用的通用置信度,并尝试将不同应用中的语音识别推理过程统一到同一框架中。

连接时序分类模型 (CTC) 是一种目前比较主流的LVCSR模型。但是由于 $\tt blank$ 的引入,使得基于 CTC的词语级别的置信度 (CM) 并不能够被直接得到,特别是最主流的针对传统基于音素似然度归一化或者基于词图后验概率的混淆网络等方法。在前面提出的标签同步解码算法的基础上,我们进一步提出了两种置信度生成算法。更细致的研究显示这种基于CTC的音素词图是得到更好性能的关键所在。
在英文Switchboard 上的大词汇连续语音识别任务显示这里提出的LSD CTC 词图置信度算法可以显著改善原先传统的基于逐帧解码算法的CTC置信度或者 HMM模型的置信度。

另一方面由于不同ASR应用之间不同的搜索空间大小和效率要求,当前业界最优的置信度及其相应的解码算法在不同应用上具有不同架构,这些不同应用包括:关键词检测,基于上下文的语音识别和大词汇连续语音识别。针对基于词图后验概率的置信度,计算量主要集中在词图部分的边缘概率计算过程。本论文中,我们提出一系列针对不同应用的通用置信度,并尝试将不同应用中的语音识别推理过程统一到同一框架中。
%
具体来说,我们提出了辅助归一化搜索空间的概念。我们尝试使用这样的搜索空间来建模所有ASR应用领域的置信度。 % and CM can be obtained in an unified framework
而针对这样做在低功耗设备上带来的挑战,我们采用基于CTC的标签同步解码\cite{Chen+2016} 来进行处理,由此带来了很大的效率改善。
最终这一统一高效的置信度框架被应用于目前主流的上述三种 ASR应用。


\section{关键词检测的序列建模和标签同步解码}
\label{chap:sum-kws}

在本论文中,我们为深度学习的声学非固定关键词的关键词检测设计了相应的序列鉴别性训练方法,同时该方法也可以应用到固定关键词的关键词检测中。关于如何对序列概率进行定义,可分为序列条件似然度和序列后验概率,这包括两种序列模型: {\em 生成式序列模型} (GSM),比如HMM, 和 {\em 鉴别式序列模型} (DSM),比如 CTC。
对于GSM,序列鉴别性训练需要在序列上使用贝叶斯公式来通过序列条件似然度得到后验概率;而DSM则可以直接使用序列后验概率。
%
对这两种框架,竞争可能性的建模都是核心难题。本论文提出两种方法来解决这一问题:隐性使用音素或半词单元的语言模型来建模,或者显性加入非关键词的标签。

具体来说,在HMM中,我们提出使用音素语言模型来对关键词的完整搜索空间进行建模。
为了加强关键词的鉴别能力,在关键词出现时候它们的梯度可以被增强。除此之外,许多神经网络结构和鉴别性训练准则也进行了详细比较。
在CTC中,非关键词建模单元被直接引入到建模当中。具体来说,加入之后的CTC模型搜索空间包括了关键词,音素边界 ($\tt blank$) 和词边界 ($\tt wb$)。 
最后我们基于前面介绍的LSD算法提出了一种高效的后处理算法,以解决音素混淆建模的问题。

总结我们的主要贡献包括:
i) 针对生成式序列模型和鉴别式序列模型的序列鉴别性训练的第一个系统研究工作
ii) 提出了一些新颖的办法来构建声学KWS的竞争可能性,以用于鉴别性训练。这些方法显著提升了关键词检测系统的性能。
iii) 基于 LSD框架提出了高效的后处理方法,以便对音素混淆性进行建模。


\section{后续工作展望}
\label{chap:sum-future}

本论文围绕基于深度序列模型的解码搜索技术展开了一系列探索和研究。进一步的后续研究方向包括:
\begin{itemize}
	\item 在本论文中,我们将并行计算的思想第一次应用到图搜索算法中。而自然语言处理中包含大量图搜索算法,将这种思路进一步推广和应用到其他任务中将会是非常有意义的研究。
	\item 在本论文中,我们基于端到端建模来简化传统语音识别中的搜索解码计算量。未来的研究包括两方面:一是进行更加端到端的建模,以进一步逐渐简化搜索解码;二是如何从深度学习的角度解释传统图搜索算法,由此提出更高效的深度学习模型的来替代图搜索算法。
	\item 在本论文中,我们研究了关键词检测系统的序列鉴别性训练。采用其他的序列训练或序列准则以进一步改善关键词检测系统是一个有意义的研究话题,特别是尝试与本论文中提到的一些端到端建模方法相结合。
	\item 最后,将本论文提出的并行解码搜索技术与标签同步解码算法相结合,并推广到所提出的通用推理和置信度框架中,得到一个速度极大幅提升,同时性能和置信度得到改善的通用系统,将会非常有意义。同时得益于速度的大幅提升和框架的简化,该方案具有进一步推广到端侧边缘计算的极大想象空间。
\end{itemize}

