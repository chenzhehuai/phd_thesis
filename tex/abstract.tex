%# -*- coding: utf-8-unix -*-
%%==================================================
%% abstract.tex for SJTU Master Thesis
%%==================================================

\begin{abstract}

语音识别中一个独有并且有趣的自然现象是声学序列和语言学序列的长度可变性,这使得语音识别需要同时建模两种序列之间的状态对齐与模式识别。
%
在训练阶段,一组带有已知标签的输入特征被提供给系统进行模型构建,{\em 序列建模}和{\em 模式识别}是前一问题的两大核心,这决定了语音识别系统可达到的识别精度的上限;而测试阶段,则基于特征序列和其它知识源如语言模型和字典进行模型{\em 搜索解码},这决定了识别速度和实际可达的识别精度。
%
近年来,深度学习模型被引入到语音识别的声学和语言建模当中替代传统分类器,显著改善了模式识别问题的精度。 基于深度学习的语音识别由于只是替换了分类器,语音识别的序列建模和搜索解码未有本质改变。
本论文围绕深度学习背景下的序列建模和搜索解码技术展开了一系列探索和研究。

本论文针对关键词检测和多说话人重叠语音信号识别这两类非传统语音识别任务提出了序列建模方案。
序列建模方法通常只在训练标准语音识别模型时进行使用,但针对其它非传统识别任务的序列建模研究并不充分,这包括:关键词检测任务和多说话人重叠语音信号识别任务等。这类任务仍然是序列预测问题,但是却没有合适的训练准则和相应的设计,来充分优化分类器的序列建模能力。
为了将序列鉴别性训练引入关键词检测任务,核心难题是设计相应的竞争可能性建模方法。本论文提出采用无词图鉴别性训练框架来解决这一问题:隐性使用音素或半词单元的语言模型来建模。
另一方面,单通道多说话人混叠语音识别也属于序列级问题, 因此序列鉴别性准则将有助于这样的序列分类问题。
我们提出了一种传统鉴别性训练技术变种,它在进行鉴别性训练的同时,也抑制输出通道上说话人跟踪错误。通过联合优化,迁移学习,序列鉴别性训练等方式,我们改善了原来语音分离、信号增强和语音识别的联合训练系统。

本论文从解码搜索并行化和降低搜索复杂度两个层面进行搜索解码算法加速。
一方面,本论文提出并行的解码搜索算法并在GPU上实现开源该套算法, 该框架可以显著加速现有推理搜索算法。
%
语音识别解码网络中多数WFST边之间并不直接相关,具有并行处理的可能性;基于GPU的并行加速也已成功应用于声学分数的计算。但是基于GPU并行计算的WFST解码并不容易实现,且现有研究存在诸多缺陷。
针对设计中的几大难点,本论文提出了如下解决方案:
本论文将维特比算法中的令牌合并操作实现为一个GPU并行计算中的原子操作,以便减少维特比束剪枝算法中同步消耗;提出了动态负载均衡的方式以更高效地进行并行计算,提高其多线程之间的利用率;重新设计了基于GPU并行计算的精确的词图生成和剪枝算法,以便充分利用GPU的性能特点。
%
在Switchboard 上实验表明,本论文所提出的方法在取得完全一致的1-best和词图质量情况下,可以得到3-15倍的加速。除此之外,如果再进行多句子的并行处理,最终的加速比将达到46倍。所提出的离线解码器对语言模型和声学模型没有特别的限制,并且可以工作在各种架构的GPU上。

另一方面,本论文基于端到端建模,系统地提出了标签同步算法,其通过一系列方法使得搜索解码过程从逐帧同步变为标签同步,从而加速解码。
当前主流的推理搜索方法是帧层面的维特比束搜索算法,该算法复杂度很高,限制了语音识别的广泛应用。本论文提出将特征层面的搜索过程改变为标签层面,使得解码速率等于标注速率,从而小于特征速率。具体来说,在标签搜索阶段,对帧层面声学模型的输出增加一步后处理过程, 得到对每个输出标签概率计算的近似值,再进行标签搜索。与传统方法相比,该方法的优势是搜索空间更小,且搜索过程被大大加速。
本论文提出的一系列通用方法在隐马尔科夫模型和连接时序分类模型上得到了验证,取得大幅度语音识别解码速度改善。

标签同步算法同时还能产生高质量的音素词图,称为CTC音素词图。基于CTC音素词图,本论文进一步探讨标签同步解码算法的一些扩展应用。这包括关键词检测,多识别任务统一置信度框架,以及端到端语音识别。

我们在关键词检测中,基于前述LSD算法得到的音素词图,提出了一套基于编辑距离的后处理算法以引入音素混淆性,使系统更加鲁棒。
%
在前面提出的标签同步解码算法的基础上,本论文进一步提出了两种置信度生成算法。更细致的研究显示这种基于CTC的音素词图是得到更好性能的关键所在。
%
本论文还提出了辅助归一化搜索空间的概念,并尝试使用这样的搜索空间来建模所有ASR应用领域的置信度。 % and CM can be obtained in an unified framework
而针对这样做在低功耗设备上带来的挑战,本论文采用基于CTC的标签同步解码来进行处理,由此带来了很大的效率改善。
%最终这一统一高效的置信度框架被应用于目前主流的多种ASR应用。
%
同时本论文还研究了将标签同步算法应用于直接建模输出序列形态学组合的端到端模型的方案。本论文使用模块化训练的思想来改善端到端模型建模,使其更易于使用外在知识源来训练每一个端到端模型的子模块。值得注意的是,模型最后需要进行联合优化,因此最终在推理搜索阶段,模型仍然工作在端到端模式下。
在实验部分,本论文提出系统一方面取得大幅度语音识别解码速度改善,另一方面在端到端建模上取得了更快和更好的模型收敛和模型准确度。


总而言之,本论文围绕深度学习背景下的序列建模和解码搜索技术展开了一系列探索和研究。本论文提出的并行解码搜索技术与标签同步解码算法相结合,可以得到一个极大幅的速度提升;
再结合所提出的通用推理搜索和置信度方案,将得到一个低功耗、高性能的通用语音识别系统。本论文不仅在主要数据集上验证了相关算法,并且开源了部分算法系统。

\keywords{语音识别,序列建模,搜索解码,并行计算,标签同步解码,置信度,序列鉴别性训练}
\end{abstract}

\begin{englishabstract}

TODO

\englishkeywords{\large TODO, TODO}
\end{englishabstract}

